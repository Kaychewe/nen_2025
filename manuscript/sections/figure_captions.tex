% Main figure legends
\textbf{Figure 1. Transcriptomic clustering distinguishes well-differentiated NETs from poorly differentiated NECs in Clinical Cohort.} (A) Sources of tumors from FCCC Cohort. Illustration shows the anatomical distribution of the 36 formalin-fixed paraffin-embedded (FFPE) tumor samples profiled using the NanoString PanCancer Pathways panel, including 21 well-differentiated (WD) neuroendocrine tumors (NETs; grades 1--3; blue) and 15 poorly differentiated (PD) neuroendocrine carcinomas (NECs; grade 3; orange). (B) Principal component analysis (PCA) plot of normalized RNA expression values demonstrates linear separation between PD NECs and WD NETs along principal component 1 (PC1), with PD samples appearing more dispersed. Each point represents one tumor sample; PD NECs are shown in orange, WD NETs in blue. Two PD NEC samples (615 and 4709) cluster closely to WD NETs whilst two PD samples (1005 and 3067) separate from PD NEC samples. (C) Volcano plot showing differentially expressed genes (DEGs) between PD and WD tumors. Genes with $|\log_2\mathrm{FC}| \ge 1$ and FDR $< 0.05$ are highlighted. Selected genes with higher expression in PD NECs (e.g., CHEK1, TTK, EZH2) and WD NETs (e.g., CACNB2, CAMK2B, RASA4, PAK3) are labeled. Upregulated genes are shown in red, downregulated genes in blue and no significant genes are shown in grey.

\textbf{Figure 2. Unsupervised hierarchical clustering identifies a third transcriptional group.} (A) Heatmap of normalized gene expression (voom + limma) for the top variably expressed genes across 36 NEN tumor samples profiled by NanoString ($\sim$800 cancer-related genes). Rows represent genes (z-score scaled), and columns represent tumors, annotated by clinical and genomic features including age, gender, race, grade, stage, Ki-67 index, primary site, and selected mutations (e.g., TP53, KRAS, PIK3CA). Hierarchical clustering reveals three major groups: Group 1 (predominantly WD NETs), Group 2 (predominantly PD NECs), and Group 3 (intermediate/mixed subtype). (B) Principal component analysis (PCA) of RNA expression by subtype clustering. Tumors from Group 3 (e.g., cases 3067 and 1005) occupy an intermediate transcriptional space between canonical WD and PD clusters, suggesting a molecularly distinct subgroup. Tumor groups are colored by clustering assignment: Group 1 (blue), Group 2 (orange), and Group 3 (purple).

\textbf{Figure 3. Immunohistochemical validation of subtype-specific markers.} (A) Representative EZH2 immunohistochemistry (IHC) staining in a PD NEC tumor shows strong nuclear expression. (B) Representative PAK3 IHC staining in a WD NET tumor shows increased cytoplasmic localization. (C--D) Boxplots summarizing IHC scores for EZH2 (C) and PAK3 (D) across FCCC NEN cohort (PD NEC, n = 15; WD NET, n = 21). EZH2 shows significantly higher expression in PD NECs (FDR-adjusted p = 9.38e-05), while PAK3 expression is significantly elevated in WD NETs (FDR-adjusted p = 2.67e-08). P-values were calculated using the two-tailed Wilcoxon rank-sum test and corrected using the Benjamini--Hochberg method.

\textbf{Figure 4. Machine learning classifiers identify discriminative transcriptomic biomarkers and subtype-specific gene programs.} (A) Receiver operating characteristic (ROC) curves showing performance of logistic regression (LR; dashed/solid green lines) and random forest classifier (RFC; dashed/solid lines) models on both training and test datasets. Curves represent classification of NEC vs NET using transcriptomic profiles; area under the curve (AUC) is annotated for each model. (B) Barplot of top ten and bottom ten genes ranked by LR model coefficients. LR coefficients are shown in blue, representing higher weight toward NEC and NET classification, respectively. (C) Barplot showing SHAP (SHapley Additive exPlanations) feature importance values derived from the RFC model. Green bars represent the relative contribution of each gene to model predictions. (D) Venn diagram showing the overlap of top-ranking features (top 50 genes) across three importance ranking methods: LR coefficient, SHAP importance, and RFC Gini importance. CAMK2B is labeled as one of the genes identified by all three approaches. (E) Expression profile of CAMK2B across adult human tissues based on RNA-seq from the GTEx Portal. Color shading represents tissue-level expression (TPM, transcripts per million). (F) Barplot of select genes upregulated in WD NETs in FCCC cohort based on DE analysis (log2FC > 1, FDR < 0.05). (G) Barplot of select genes upregulated in PD NECs under the same criteria. (H) Pathway enrichment analysis of genes with positive coefficients from the LR model. The most significantly enriched pathway is WP179 Cell Cycle pathway (p = 8.49 $\times$ 10$^{-31}$) based on Reactome similarity scoring. Positive LR coefficients and PD-associated SHAP features are marked by open triangles, respectively.

\textbf{Figure 5. Transcriptomic similarity and primary site inference based on gene expression.} (A) Principal component analysis (PCA) of RNA expression across NEN tumor samples, colored by inferred primary tumor site. Sites were assigned using a K-nearest neighbors (KNN) classifier trained on samples with known origins. Shape denotes histological subtype: squares indicate poorly differentiated (PD) NECs, circles indicate well-differentiated (WD) NETs. Fill denotes data partition: closed black shapes represent test set tumors (n = 6), open shapes represent training set tumors (n = 30). (B) Quantile--quantile (QQ) plot comparing expression ranks between sample SC14-4289\_PD and its most similar neighbor SC14-299\_PD based on transcriptomic similarity. (C) Heatmap of pairwise Pearson correlation coefficients (r) for all tumor samples; correlation strength is visualized on a continuous color scale. (D) QQ plot for the tumor pair with the lowest transcriptomic correlation (CO14-3067\_PD vs SO11-2366\_WD). (E) QQ plot for the tumor pair with the highest correlation (SO13-5018\_WD vs SC14-3857\_WD).

\textbf{Figure 6. DNA methylation--based biomarkers distinguish NETs from NECs and reflect transcriptional and epigenetic subtype-specific programs.} (A) Principal component analysis (PCA) of $\beta$-values from Infinium MethylationEPIC arrays (GSE211483; n = 33 lung NEN samples) reveal separation of normal lung (green), well-differentiated NETs (purple), and poorly differentiated NECs (orange) along principal components PC1 and PC2. (B) Heatmap of the top 20 differentially methylated CpG sites (FDR < 0.05, $|\Delta\beta| \ge 0.20$) across lung NEN samples. Color indicates methylation $\beta$-value (blue = hypomethylated, red = hypermethylated); samples are grouped by tissue type. (C--F) Violin plots of candidate CpG biomarkers showing methylation levels across normal lung (green), NET (purple), and NEC (orange) groups. (C) SFN (cg07786675, TSS200/1stExon) shows NEC-specific hypomethylation. (D) FANCA (cg08211068, TSS1500/5\textquotesingle{}UTR) shows NEC-specific hypomethylation. (E) CAMK2B (cg05610097, gene body/enhancer) exhibits NET-specific hypomethylation. (F) FGF14 (cg22579071, TSS200/promoter) shows elevated methylation in NETs relative to NECs and normals. (G) Summary table of significant CpGs for SFN, FANCA, CAMK2B, and FGF14 across two contrasts: NET--Normal and NEC--NET. Numbers indicate the count and percentage of probes meeting FDR < 0.05 and $|\Delta\beta| \ge 0.20$ criteria. (H) Stacked bar plots comparing proportions of hyper- and hypomethylated CpGs in key genomic categories (CGI promoters, CGIs, promoters) across NET--Normal and NEC--NET contrasts. Red = hypermethylated; blue = hypomethylated. P-values from Fisher's exact test. (I) Stemness index scores (pcgtAge) derived from Polycomb-target CpGs show stepwise increases across normal lung, NETs, and NECs, with NECs showing the highest epigenetic stemness. (J) Matched methylation and expression profiles for MKI67: Left panel shows log2 RNA expression (GSE211486); right panel shows $\beta$-values for cg22260973. MKI67 is significantly upregulated ($\Delta$expr = 1.17, FDR = 0.0066, p = 1.72e-04) and hypomethylated ($\Delta\beta$ = -0.231, FDR = 0.0454, p = 8.09e-04) in NECs. (K) GRIA3 RNA expression across NETs and NECs shows NEC-specific downregulation (p.adj = 6.36$\times$10$^{-4}$). (L) PRC1 RNA expression shows NEC-enriched expression (p.adj = 6.36$\times$10$^{-4}$). (M) EZH2 RNA expression is significantly elevated in NECs (p.adj = 1.71$\times$10$^{-3}$). (N) PAK3 expression is significantly reduced in NECs (p.adj = 0.021). Expression values are derived from RNA-seq (GSE211486), and $\beta$-values are from methylation array data (GSE211483). All p-values reflect Wilcoxon rank-sum tests with Benjamini--Hochberg FDR correction.
