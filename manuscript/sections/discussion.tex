\subsection{\texorpdfstring{Discussion}{Discussion}}\label{discussion}

We show that transcriptomic profiles can support classification of NEN
subtypes using interpretable machine learning. Logistic regression and
random forest models achieved \textasciitilde87--93\% accuracy in
cross-validation for PD-NEC versus WD-NET, and performed as expected on
an external WD-NET dataset (GSE118014). This exceeds earlier limited
panel approaches (e.g., \textasciitilde71--78\% accuracy in small
intestinal NETs \cite{I2009Predicting}), though differences in cohorts
and assay design preclude direct benchmarking. We also observed a small
cluster with mixed features; given its size, this should be viewed as
hypothesis-generating rather than a defined subtype.

Together, these results support a differentiation-linked axis. WD-NETs
retain neuroendocrine lineage programs, whereas PD-NECs adopt a
proliferative, epigenetically repressed state. The axis is detectable
across transcriptomic and epigenetic layers and appears largely
independent of tissue of origin.

PD-NECs were characterized by upregulation of cell-cycle and replication
stress regulators (\emph{SFN}, \emph{CHEK1}, \emph{E2F1}, \emph{CDC6},
\emph{TTK}), consistent with high proliferation \cite{M2022Expanding}.
WD-NETs showed enrichment of neuroendocrine lineage and synaptic genes
(\emph{CAMK2B}, \emph{GRIA3}, \emph{PAK3}, \emph{FGF14}), aligning with a
more differentiated phenotype. These results are concordant with
observations that high-grade NECs share programs with small-cell lung
cancer \cite{M2022Expanding}. While promising, clinical deployment would
require harmonized assays, calibrated thresholds, and prospective
validation.

Many PD-NEC-associated features regulate DNA replication, checkpoint
control, and chromatin state, suggesting a shared proliferative program,
while WD-NET features reflect lineage and neuronal signaling programs.

We validated subtype-specific protein expression by IHC. \emph{EZH2} and
\emph{H3K27me3} were higher in PD-NECs, while \emph{PAK3} was higher in
WD-NETs. External methylation data from lung NENs supported an
epigenetic signature of dedifferentiation in NECs (promoter
hypermethylation, higher pcgtAge, and methylation--expression shifts at
\emph{MKI67} and \emph{SFN}). Recent reports also show broad expression
differences between G3 NETs and NECs \cite{V2024Comprehensive}. WD-NET
markers in our study clustered in neuronal/secretory programs
(\emph{CAMK2B}, \emph{GRIA3}, \emph{BAIAP3}, \emph{CACNA1D}), suggesting
candidate biomarkers, though functional validation is needed to link
these genes to NET biology.

To date, WD-NETs and PD-NECs have been considered distinct entities with
different cells-of-origin and genetic drivers
\cite{NM2024Progression}. Genomic
studies supporting this distinction highlight that WD PanNETs frequently
harbor \emph{MEN1}, \emph{DAXX}/\emph{ATRX}, and \emph{mTOR} pathway
mutations, whereas PD-NECs (in pancreas and elsewhere) often contain
disruptive \emph{TP53} and \emph{RB1} alterations
\cite{NM2024Progression}. When
examining mutation patterns we find similar trends where \emph{TP53}
mutation occurring at a higher frequency compared (Fig. 2a). A recent
2020 methylation analysis of pancreatic NENs by Simon et al. showed that
PanNECs form a completely separate epigenetic cluster from PanNETs
(including G3 well-differentiated NETs), implying an exocrine lineage
origin for PanNECs as opposed to an endocrine islet cell origin for
PanNETs \cite{T2022DNA}. This
``cell-of-origin'' divergence is further reflected in the
transcriptomes, where we observed that NET samples (even high-grade)
were uniformly classified by our model as NET-like, distinct from NECs.
Nevertheless, the spectrum from NET to NEC may not be entirely clear.
Rare cases of tumor progression from a low-grade NET to a high-grade,
NEC-like phenotype have been documented. For instance, Joseph \emph{et
al.} (2024) reported a series of G1/G2 NETs that evolved into G3
neoplasms with acquired \emph{TP53}/\emph{RB1} co-mutations, blurring
the line between G3 NET and true NEC
\cite{NM2024Progression}. In their cohort,
some progressed tumors even retained well-differentiated morphology
despite harboring genetic hallmarks of NEC, illustrating how
morphological and molecular changes can decouple in the evolution of
NENs. Such cases are exceedingly uncommon (progression from NET to NEC
is considered very rare
\cite{NM2024Progression}, but they highlight
that subtype transitions, while infrequent, are possible. More commonly,
what appears as ``NET to NEC'' transformation in the clinic may turn out
to be a misclassification or a mixed tumor scenario. For example, in
gastroenteropancreatic tumors, one can encounter mixed
neuroendocrine-nonneuroendocrine neoplasms (MiNENs) where a NET and a
carcinoma co-exist \cite{M2022Expanding}.
Additionally, outside the GI/Lung context, some non-neuroendocrine
carcinomas can transdifferentiate into a neuroendocrine phenotype under
therapy pressure -- a phenomenon known as neuroendocrine lineage
plasticity \cite{A2023Role}. This is
well documented in prostate cancer treated with androgen blockade, where
emergent neuroendocrine prostate cancer (NEPC) shares molecular programs
with small-cell lung cancer
\cite{P2021Subtype}.

The existence of NE lineage plasticity in other epithelial cancers
suggests that an epithelial tumor cell can switch on a neuroendocrine
program (often via epigenetic reprogramming
\cite{520000Colapietra}, effectively
\emph{mimicking} a PD-NEC. In the context of NENs, however, most PD-NECs
appear to arise \emph{de novo} rather than from pre-existing NETs
\cite{M2022Expanding}. Our data support
this paradigm where the clear bifurcation in gene expression and DNA
methylation profiles between WD and PD classes implies separate
evolutionary trajectories. Going forward, it will be important to
explore whether the few bona fide NET-to-NEC progression cases exhibit
intermediate ``warning'' molecular signatures -- potentially something
our classifier (or its underlying feature genes) could detect.
Monitoring high-grade NETs for acquisition of NEC-like molecular traits
(such as a surge in cell-cycle gene expression or loss of endocrine
markers) might help flag patients at risk of aggressive transformation.

Several limitations should be noted. The discovery cohort is modest in
size and heterogeneous in primary site, and the models were trained on a
targeted NanoString panel rather than whole-transcriptome profiling.
While we used cross-validation and external datasets, the analysis
remains retrospective and subject to selection and batch effects.
Larger, prospectively collected cohorts with harmonized assays will be
required to define clinically actionable thresholds, evaluate
generalizability across sites, and test whether classifier-guided
decisions improve outcomes.

More broadly, this framework illustrates how interpretable
transcriptomic models can expose conserved cellular states that cut
across anatomical context. Similar differentiation--proliferation axes
have been reported in other epithelial cancers with lineage plasticity,
suggesting these approaches may generalize beyond NENs.
