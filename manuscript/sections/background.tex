\subsection{\texorpdfstring{Background}{Background}}\label{background-1}

Neuroendocrine neoplasms (NENs) arise from specialized epithelial cells
that secrete amine or peptide hormones in response to neural and
endocrine stimuli. These cells are most abundant in the gastrointestinal
(GI) tract, pancreas, and lungs but occur across many organ systems
\cite{AM2016New,A2017Wholegenome,SV2019Integrative,C2017Distinctive,A2019Genetics,A2020Comprehensive}.
Although NENs remain less common than many solid tumors, incidence has
risen substantially. In the United States, rates increased from 1.09 per
100,000 in 1973 to 6.98 per 100,000 in 2012 (6.4-fold)
\cite{Das2021Epidemiology}, with parallel rises in the United Kingdom and Taiwan
\cite{BE2022Incidence}.

Accurate histological classification is essential for prognosis and
therapy yet remains challenging because key features overlap across
subtypes \cite{JY2022Making}. Differentiation status, assessed by
morphology and proliferation markers, is the strongest prognostic
determinant \cite{MH2021Neuroendocrine}, but tumors with similar
histology can follow divergent clinical courses. PD-NECs are highly
aggressive with rapid progression and limited treatment durability
\cite{150000Klimstra}; pancreatic PD-NECs have median overall survival of
\textasciitilde11--12 months \cite{O2014Poorly,JM2005Metastatic}. In
contrast, WD-NETs are typically indolent with median survival of 5 to
\textgreater10 years depending on site and grade; for example, pancreatic
WD-NETs show 8--12 years and small-intestinal NETs often exceed 65--75\%
5-year survival \cite{A2017Trends,BE2022Incidence,MH2021Neuroendocrine}.

The World Health Organization (WHO) framework \cite{LH2016A,ID2020The}
relies on histomorphology and proliferation thresholds (mitotic count
\textgreater20 per 2mm\textsuperscript{2} and Ki-67 \textgreater20\%) to
classify PD-NECs as high-grade tumors \cite{MH2021Neuroendocrine}. While
clinically essential, this scheme has limitations. WD-NETs span a broad
grade spectrum (G1--G3), and high-grade WD-NETs can be morphologically
indistinguishable from PD-NECs. Immunohistochemical markers such as
chromogranin A and synaptophysin, and aberrant p53/RB1 patterns, are
helpful but not definitive and frequently overlap in high-grade tumors
\cite{LH2016A,MS2013The}.

A recent multi-institutional study reported poor interobserver agreement
in high-grade NEN classification, with experts failing to reach
consensus in nearly two-thirds of cases \cite{LH2016A}. This highlights
the limitations of morphology-only approaches and motivates quantitative
molecular classifiers.

Next-generation sequencing (NGS) has advanced understanding of the
molecular landscape \cite{150000Klimstra,200000Couvelard}, but molecular
profiling has also revealed overlap between WD-NETs and PD-NECs. Some
WD-NETs exhibit high proliferation and TP53 alterations typical of
PD-NECs, limiting diagnostic specificity of these markers
\cite{LH2016A,NM2024Progression}. PD-NEC drivers remain incompletely
characterized due to rarity and substantial intra- and inter-tumoral
heterogeneity \cite{J2021The,M2022Expanding,V2024Comprehensive}. NENs also
have low tumor mutational burden (TMB) compared with many solid tumors
\cite{A2020Comprehensive,J2021The}, constraining genomics-only approaches.
In a cohort of 85 metastatic/advanced NENs, only 16 (18.82\%) were
PD-NECs \cite{J2021The}, limiting power to define robust subtype drivers.

Several molecular studies have sought to refine NEN classification
beyond morphology. Protein expression analyses show loss of
\emph{DAXX}/\emph{ATRX} in WD-NETs and aberrant p53/\emph{RB1} in PD-NECs
\cite{MS2013The}. Epigenetic profiling further supports distinct DNA
methylation landscapes \cite{T2022DNA}, but many studies are limited to
pancreatic NENs or emphasize tissue-of-origin rather than
histologic differentiation. Large methylation studies can infer tissue
site across organs \cite{WM2021Genome,A2020Distinct,A2020Epigenetic}, yet
often do not resolve the PD-NEC versus WD-NET distinction. Existing
approaches can be constrained by anatomical specificity, reliance on
single markers, or unsupervised clustering with limited clinical
interpretability and technical variability. Transcriptomic profiling
offers a complementary framework by capturing active cellular programs
(lineage identity, proliferation, stress responses). Whether
differentiation state represents a conserved, tissue-agnostic molecular
axis in NENs remains unclear. Transcriptomic signatures can also be
confounded by tissue-of-origin and microenvironmental composition,
underscoring the need for interpretable models and orthogonal
validation.

To address these limitations, we developed an interpretable
transcriptome-based machine learning framework to distinguish PD-NECs
from WD-NETs across diverse anatomical sites. Using targeted
transcriptomic profiling, we trained supervised classifiers (random
forest and logistic regression) that explicitly model differentiation
while enabling direct interrogation of feature importance. This approach
identified a conserved molecular axis: activation of cell-cycle,
replication stress, and DNA damage response programs in PD-NECs versus
preservation of neuroendocrine lineage and neuronal signaling programs
in WD-NETs. We validated these signatures using orthogonal
immunohistochemical markers and independent DNA methylation datasets,
demonstrating concordant stratification and increased epigenetic
stemness in PD-NECs. Together, this framework provides a quantitative,
biologically interpretable basis for resolving histologic ambiguity and
highlights transcriptomic state as a robust indicator of
differentiation-related aggressiveness.
