\subsection{\texorpdfstring{Abstract
}{Abstract }}\label{abstract}

\subsubsection{Background}
Neuroendocrine neoplasms (NENs) comprise biologically heterogeneous tumors whose classification into well-differentiated neuroendocrine tumors (WD-NETs) and poorly differentiated neuroendocrine carcinomas (PD-NECs) remains clinically challenging. Morphology and proliferation indices alone often fail to resolve high-grade cases, motivating molecular frameworks that capture differentiation state across tissues.
\subsubsection{Results}
We profiled 36 FFPE NENs from 11 anatomical sites using a targeted 784-gene transcriptomic panel and developed interpretable machine-learning classifiers to model histologic differentiation. Across cross-validation, models achieved stable discrimination between WD-NETs and PD-NECs (87--93\% accuracy). Feature attribution identified a conserved molecular axis separating lineage-preserved WD-NETs from proliferative, replication stress--enriched PD-NECs. WD-NETs retained neuroendocrine and neuronal signaling programs, whereas PD-NECs exhibited activation of cell-cycle, DNA damage response, and chromatin-regulatory pathways.

Orthogonal validation confirmed these findings at multiple levels: subtype-specific protein expression by immunohistochemistry (e.g., EZH2, PAK3) and independent DNA methylation profiling from an external lung NEN cohort demonstrated concordant stratification and increased epigenetic stemness in PD-NECs. Together, transcriptomic and epigenetic analyses reveal a conserved differentiation state that transcends anatomical origin.
\subsubsection{Conclusions}
Integrated multi-omic profiling identifies a tissue-agnostic differentiation axis that robustly stratifies neuroendocrine neoplasms and provides biologically interpretable biomarkers for resolving histologic ambiguity. These findings establish differentiation state as a unifying molecular framework for NEN classification and suggest a foundation for future diagnostic and therapeutic development.
