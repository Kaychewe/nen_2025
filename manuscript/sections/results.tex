\subsection{\texorpdfstring{Results}{Results}}\label{results-1}

\subsubsection{\texorpdfstring{Principal component analysis reveals a conserved differentiation axis separating PD-NEC and WD-NET
}{Principal component analysis reveals a conserved differentiation axis separating PD-NEC and WD-NET }}\label{principal-component-analysis-reveals-transcriptomic-separation-of-pd-nec-and-wd-net}

We profiled 36 neuroendocrine neoplasms (NENs) from the Fox Chase Cancer Center (FCCC) spanning 11 primary anatomical sites (esophagus, lung, breast, stomach, pancreas, small bowel, cecum, colon, rectum, ileum) and an additional five tumors of unknown origin (Fig. 1A; Supplementary Table S1-2). Independent pathological review classified 21 tumors (58\%) as well-differentiated neuroendocrine tumors (WD-NETs) and 15 (42\%) as poorly differentiated neuroendocrine carcinomas (PD-NECs). Transcriptomic profiling was performed using the NanoString nCounter PanCancer Immune Profiling Panel with customization (784 genes), capturing cell-cycle, DNA damage response, chromatin regulation, signaling, and immune-related programs.\cite{Cesano2015PanCancer,NanoStringPanCancerImmunePanel}

To determine whether global transcriptional variation reflected differentiation state rather than tissue of origin, we performed principal component analysis (PCA) on normalized log-transformed expression data. Unsupervised PCA revealed clear separation of PD-NECs and WD-NETs along principal component 1 (Fig. 1B). Permutational multivariate analysis of variance (PERMANOVA) confirmed that histologic subtype explained a significant proportion of transcriptomic variance (R\textsuperscript{2} = 0.17, F = 6.9, p = 0.001), supporting differentiation status as a dominant molecular dimension across anatomically diverse tumors.

Importantly, separation was observed despite substantial heterogeneity in primary site, suggesting the presence of a conserved, tissue-agnostic transcriptional program linked to differentiation state. Prior transcriptomic studies have identified subtype-associated gene expression differences within restricted anatomical contexts; however, whether a unified differentiation-linked program persists across tissues has remained unclear.\cite{I2009Predicting,CS2018ATRX,V2022Identification,V2024Comprehensive} Our findings indicate that conservation is detectable even within a heterogeneous multi-site cohort.

To define genes contributing to this separation, differential expression analysis was performed (\textbar log2FC\textbar{} \textgreater{} 1, FDR \textless{} 0.05). PD-NECs exhibited increased expression of genes involved in replication licensing and checkpoint control, including \emph{CDC6}, \emph{CHEK1}, \emph{SFN}, and the epigenetic regulator \emph{EZH2} (Fig. 1C). These genes collectively reflect a replication-competent, checkpoint-activated state consistent with high mitotic activity and replication stress adaptation. In contrast, WD-NETs showed higher expression of genes associated with neuronal signaling and regulated secretory function, including \emph{PAK3}, \emph{CAMK2B}, \emph{GRIA3}, and \emph{RASA4}, indicating preservation of neuroendocrine lineage identity.

Pathway enrichment analysis mirrored these findings (Fig. 1D). PD-NECs were enriched for cell-cycle progression, homologous recombination, and DNA damage response pathways, whereas WD-NETs were enriched for neuronal, synaptic, and cell--cell signaling programs. Together, these results support a model in which PD-NECs and WD-NETs occupy distinct positions along a conserved differentiation axis characterized by proliferative reprogramming versus lineage preservation.

Given the modest cohort size, these findings should be interpreted as strong but preliminary signals. Nevertheless, the consistency of subtype separation across anatomically diverse tumors suggests that differentiation state represents a fundamental organizing principle of NEN biology.

\subsubsection{Unsupervised clustering reveals molecular substructure and an intermediate differentiation state}\label{unsupervised-hierarchical-clustering-nen-molecular-subgroup}

To assess whether transcriptomic variation resolved tumors into molecular subgroups, we performed unsupervised hierarchical clustering of normalized expression profiles. Clustering identified three groups (Fig. 2A). Group 1 (n = 20) was composed predominantly of WD-NETs and was characterized by elevated expression of genes associated with neuroendocrine lineage preservation and calcium-dependent signaling, including \emph{CACNB2}, \emph{ZBTB16}, \emph{GRIA3}, and \emph{WNT4} (Supplementary Fig. 1a--d). These features suggest a transcriptional program consistent with regulated secretory identity and neuronal differentiation. Group 2 (n = 12) was enriched for PD-NECs and exhibited increased expression of genes involved in cell-cycle regulation, chromatin remodeling, and oncogenic signaling, including \emph{CDKN2A}, \emph{EZH2}, \emph{MYB}, \emph{IL8}, \emph{CTNNB1}, and \emph{NOTCH1} (Supplementary Fig. 1e--h). The coordinated upregulation of cell-cycle and epigenetic regulators suggests a proliferative and transcriptionally reprogrammed state aligned with poorly differentiated carcinoma biology. Group 3 (n = 4) comprised tumors of mixed histology (2 WD-NETs, 2 PD-NECs) and displayed attenuated expression of both Group 1 and Group 2 marker sets, along with increased \emph{GATA3}. Two cases harbored \emph{TP53} mutations without concurrent \emph{RB1} alteration. These tumors were more dispersed in PCA space (Fig. 2B), indicating greater transcriptional heterogeneity. 

\subsubsection{Orthogonal validation of differentiation-linked markers by immunohistochemistry}\label{immunohistochemistry-biomarkers-ezh2-pd-nec-and-pak3-wd-net}

To validate transcriptomic markers at the protein-level we performed quantitative immunohistochemistry (IHC) on an independent tissue microarray (TMA) using a non-overlapping discovery set. Based on differential expression and biological relevance, we selected \emph{EZH2} as a PD-NEC--associated marker and \emph{PAK3} as a WD-NET--associated marker.

Consistent with transcriptomic findings, PD-NECs demonstrated significantly higher nuclear expression of EZH2 and its associated repressive chromatin mark H3K27me3 (Fig. 3A--B; Supplementary Fig. 2). EZH2 was selected due to robust mRNA upregulation (fold change = 4.2, adjusted p = 9.38 \times 10\textsuperscript{\textminus}5) and its established role in Polycomb-mediated epigenetic repression and proliferative signaling. Increased H3K27me3 further supports the presence of a chromatin-repressed, dedifferentiated transcriptional state in PD-NECs.

In contrast, WD-NETs exhibited significantly higher PAK3 expression (fold change = 7.7, adjusted p = 2.67 \times 10\textsuperscript{\textminus}8), consistent with retention of neuronal and neuroendocrine lineage programs. Nuclear H-scores (0--300) confirmed subtype-specific protein expression patterns, demonstrating concordance between RNA and protein levels. Antibody specificity for EZH2 antibody (CS5246) was validated by depletion experiments in retinal pigment epithelial (RPE) cells, which abrogated signal intensity (Supplementary Fig. 3A--C). 

\subsubsection{Interpretable machine learning resolves a conserved NEN differentiation}\label{interpretable-classifiers-identify-differentiation-linked-transcriptional-programs}

Distinguishing poorly differentiated NECs from high-grade WD-NETs remains a major diagnostic challenge, particularly in morphologically ambiguous tumors. To test whether transcriptional state alone could robustly encode histologic identity, we trained two interpretable classifiers---logistic regression (LR) and random forest (RF)---on normalized NanoString expression profiles (\textasciitilde784 genes).

Both models achieved stable discrimination under stratified cross-validation (AUC 87--93\%), potentially indicating that differentiation state is strongly encoded within the transcriptome (Fig. 4A). Importantly, these models were not black boxes: feature weights revealed coherent biological programs underlying classification.

As an external sanity check, we applied the trained classifier to an independent pancreatic WD-NET dataset (GSE118014). All samples were predicted as WD-NET, consistent with known histology, supporting model specificity rather than overfitting.

\subsubsection{Convergent feature attribution identifies core subtype-defining genes}\label{transcriptomic-biomarkers-for-histological-differentiation-from-interpretable-machine-learning}
To further interpret classifier outputs at the individual sample level, we computed SHAP (Shapley Additive exPlanations) values from the trained random forest model. SHAP values estimate the marginal contribution of each feature to a specific prediction, offering interpretable insights into the decision boundaries for PD NECs and WD NETs (Fig. 4C). Aggregated SHAP importances across PD NEC predictions revealed that \emph{GNG7} had the highest contribution (mean SHAP = 0.0056), followed by \emph{FANCA}, \emph{HIST1H3B}, \emph{CDC6}, \emph{TTK}, and \emph{SFN}, which includes genes involved in DNA repair, chromatin regulation, and mitotic control.

Aggregated SHAP importances across PD NEC predictions revealed that \emph{GNG7} had the highest contribution (mean SHAP = 0.0056), followed by \emph{FANCA}, \emph{HIST1H3B}, \emph{CDC6}, \emph{TTK}, and \emph{SFN}, which includes genes involved in DNA repair, chromatin regulation, and mitotic control. \emph{FANCA} and \emph{CDC6} are essential for replication fork stability, while \emph{TTK} encodes a dual-specificity kinase required for centrosome duplication and chromosome alignment \cite{H2020TTK}. \emph{HIST1H3B}, a histone variant, and \emph{SFN}, a 14-3-3 checkpoint protein, also showed elevated importance, reinforcing their role in PD NEC pathogenesis. Notably, several of these SHAP-prioritized genes (\emph{FANCA}, \emph{HIST1H3B}, \emph{CDC6}, \emph{SFN}) overlapped with top positive-weight features from logistic regression (Fig. 4B), underscoring their consistency across models. Conversely, features with high negative SHAP values included \emph{CAMK2B} (–0.0058) and \emph{PAK3} (–0.0027), further supporting their predictive value for WD NETs. 

To identify robust biomarkers across modeling strategies, we intersected feature rankings from three independent metrics (1) Logistic regression coefficients ≥ |0.04|, (2) Random forest Gini importance ≥ 0.001 and (3) SHAP importance for PD NEC class ≥ 0.001. Eight genes met all three criteria: \emph{CACNB2}, \emph{PAK3}, \emph{FANCA}, \emph{CDC6}, \emph{HIST1H3B}, \emph{BRIP1}, \emph{DKK1}, and \emph{CAMK2B} (Fig. 4D). These genes span DNA repair (e.g., \emph{BRIP1}, \emph{FANCA}), chromatin regulation (\emph{HIST1H3B}), cell cycle checkpoints (\emph{CDC6}), and neuroendocrine signaling (\emph{CAMK2B}, \emph{PAK3}). Full statistics are provided in Supplementary Tables 5–6. 


\subsubsection{Pathway architecture reveals a replication--lineage polarity}\label{pd-nec-genes-are-enriched-in-cell-cycle-pathways-while-wd-net-genes-are-brain-function-related}

Pathway analysis of PD-weighted genes revealed significant enrichment of cell-cycle programs (WP179; FDR \textasciitilde{}10\textsuperscript{\textminus}31), G1/S and G2/M transitions, ATR checkpoint signaling, Fanconi anemia, PI3K--Akt--mTOR signaling, and RB pathway dysregulation (Table 1; Fig. 4H). The core module centers on \emph{CHEK1}, \emph{SFN}, \emph{CDC6}, \emph{E2F1}, \emph{TTK}, and \emph{MCM} family members, forming a tightly connected replication stress and mitotic checkpoint axis.

In contrast, WD-NET--associated genes mapped to neuronal development and synaptic signaling programs, including calcium channel subunits (\emph{CACNB2}, \emph{CACNA1D}) and glutamatergic signaling genes (\emph{GRIA3}). Together, these results define a replication-driven PD-NEC state versus a lineage-preserved WD-NET state, paralleling programs observed in small-cell lung carcinoma and other high-grade neuroendocrine carcinomas.

\subsubsection{Transcriptome-guided inference of tumor origin}\label{primary-site-inference-of-the-unknown-tumors-from-transcriptome-profiles}

NENs frequently lack canonical driver mutations, limiting the utility of genomic data including in determining tumor site of origin \cite{V2022Identification}. For instance, whole exome sequencing of 48 small intestinal NENs revealed a median of just 0.1 somatic SNVs per megabase (range: 0-0.59) \cite{Banck2013Genomic}. Similarly, pancreatic NENs exhibit lower TMB (0.82/Mb) compared to their exocrine counterparts, with mutations in MEN1, DAXX, and ATRX paradoxically associated with favorable outcomes \cite{Y2011DAXXATRX}. As a result, transcriptional profiling may offer a more robust strategy for inferring tumor origin and classification. Using Pearson correlation--based KNN (K=1), we evaluated five tumors lacking annotated primary sites (Fig. 5).

In our study,  five samples lacked known primary sites. Four were poorly differentiated (PD_0001, PD_0009, PD_0014, PD_0015) and one well-differentiated (WD_0021). PD\_0009 showed strong similarity to a small bowel NEC (\emph{r} = 0.92). PD\_0014 and PD\_0015 matched pancreatic NEC (\emph{r} \textgreater{} 0.9). PD\_0001 demonstrated weaker similarity (\emph{r} = 0.48), highlighting limitations of nearest-neighbor approaches in heterogeneous NECs (Fig. 5B-E). Correlations were generally higher among WD-NETs (e.g., \emph{r} = 0.97 between WD\_0006 and WD\_0017), consistent with their tighter clustering in PCA space (Fig. 2B). While exploratory, these findings suggest transcriptomic proximity may assist in origin inference when appropriate reference cohorts exist.

\subsubsection{Orthogonal validation via DNA methylation profiling}\label{validation-of-transcriptomic-subtypes-using-dna-methylation-profiling}

To test whether the transcriptional differentiation axis is reflected at the epigenetic level, we analyzed an independent lung NEN cohort profiled by EPIC methylation array (GSE211483). Principal component analysis of methylation  \(\beta\)-values showed seperation of PD-NECs from WD-NETs, with PC1 explaining \textasciitilde19.2\% of variance (Fig. 6A). This mirrors transcriptomic separation (Fig. 1B), suggesting cross-modal coherence.

\textbf{PD-NECs exhibit promoter hypermethylation and Polycomb-associated repression.} Differential methylation analysis (FDR \textless{} 0.05, |\(\Delta\beta\)| \textgreater{} 0.2) identified \textgreater{}2,000 DMPs enriched in PD-NECs. Promoter- and CpG island--associated hypermethylation was prominent in NECs, including developmental loci (e.g., \emph{HOXA2}, \emph{HOXA3}), and showed significant promoter/island bias (Fisher's exact p \textless{} 10\textsuperscript{\textminus}3). This pattern is consistent with a CpG island hypermethylator-like phenotype.

\textbf{Methylation--expression concordance reinforces classifier biology.} Several classifier-prioritized genes showed methylation--expression coupling: \emph{MKI67} displayed higher expression and promoter hypomethylation in NECs (\(\Delta\)Expr = 1.17; \(\Delta\beta = -0.231\)), \emph{SFN} and \emph{FANCA} were hypomethylated in NECs, and \emph{CAMK2B}/\emph{FGF14} showed NET-biased methylation patterns (Fig. 6C--F, J--N). Wilcoxon tests confirmed significance for multiple loci (e.g., \emph{EZH2} p.adj = 0.0017; \emph{GRIA3} p.adj = 0.0006).

\textbf{Increased epigenetic stemness in PD-NECs.} pcgtAge scores were highest in NECs, intermediate in NETs, and lowest in normal lung (Fig. 6I), paralleling the transcriptional proliferation axis and indicating a more dedifferentiated epigenetic state.
