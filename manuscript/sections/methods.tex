\subsection{\texorpdfstring{Methods}{Methods}}\label{methods}

\subsubsection{Study design and
oversight}\label{study-design-and-oversight}

This was a single-center, retrospective study conducted at Fox Chase
Cancer Center (FCCC) to develop and validate molecular classifiers of
neuroendocrine neoplasm (NEN) histology. Formalin-fixed,
paraffin-embedded (FFPE) tumor specimens were profiled by targeted
transcriptomics; selected markers were validated by immunohistochemistry
(IHC) on tissue microarrays (TMAs). External orthogonal validation was
performed using public DNA methylation datasets. All procedures were
approved by the FCCC Institutional Review Board, and written informed
consent was obtained from all participants. The study was exploratory in
nature; sample size was determined by tissue availability for the rare
poorly differentiated (PD) NEC subtype rather than a priori power
calculations.

\subsubsection{Cohort and tissue
microarrays}\label{cohort-and-tissue-microarrays}

Tumor specimens were obtained from patients with well-differentiated
(WD) neuroendocrine tumors (NETs) and PD neuroendocrine carcinomas
(NECs) from multiple primary sites. Clinicopathologic data were
de-identified prior to analysis. TMAs were constructed from 36 FFPE
blocks (21 WD, 15 PD). Hematoxylin-and-eosin--stained sections were
reviewed by a board-certified pathologist to mark representative tumor
regions. Duplicate cores (diameter 1 mm) were taken per case using a
precision arrayer (Beecher Instruments, Silver Spring, MD, USA) and
placed on separate recipient blocks to mitigate positional bias. Tissue
had been fixed in 10\% phosphate-buffered formaldehyde (24-48 h),
processed, and embedded per standard protocols. Each TMA core was scored
individually; duplicate core values were averaged per case.

\subsubsection{Immunohistochemistry and quantitative image
analysis}\label{immunohistochemistry-and-quantitative-image-analysis}

IHC was performed on 5 µm TMA sections. Slides were deparaffinized,
rehydrated, and subjected to heat-induced epitope retrieval in 0.01 M
citrate buffer (pH 6.0) for EZH2, H3K27me3, PAK3, and pCHK1, or EDTA
buffer (for pSTAT3). Endogenous peroxidase activity was quenched in 3\%
H₂O₂. Sections were incubated overnight at 4 °C with primary antibodies:
EZH2 (D2C9, rabbit, 1:50; Cell Signaling Technology, \#5246),
Tri-methyl-Histone H3 (Lys27) (C36B11, rabbit, 1:50; Cell Signaling,
\#9733), PAK3 (A-20, rabbit, 1:500; Invitrogen, PA5-15118), phospho-Chk1
(Ser345, rabbit, 1:500; Thermo Fisher, PA5-34625), and phospho-STAT3
(Tyr705, D3A7, rabbit, 1:30; Cell Signaling, \#9145). Detection used the
EnVision+ polymer system (Dako/Agilent) with 3,3′-diaminobenzidine (DAB)
chromogen. Slides were counterstained with hematoxylin, dehydrated,
cleared in xylene, and mounted. Negative controls used matched isotype
control IgG.

Whole-slide images were acquired on a PerkinElmer CRi Vectra 2 platform.
Multispectral images were unmixed and analyzed in inForm (Akoya
Biosciences) using a trainable workflow: (i) tissue segmentation to
identify neuroendocrine tumor regions and (ii) cell segmentation to
delineate hematoxylin-positive nuclei and DAB-positive subcellular
compartments. Staining intensity was binned (0/1+/2+/3+), and H-scores
were computed as 1×(\%1+) + 2×(\%2+) + 3×(\%3+) (range 0--300). Nuclear
and cytoplasmic H-scores were derived as appropriate. Antibody
specificity for EZH2 was confirmed by loss of signal following EZH2
depletion in retinal pigment epithelial (RPE) cells.

\subsubsection{Transcriptomic profiling
(NanoString)}\label{transcriptomic-profiling-nanostring}

FFPE sections were reviewed to enrich for invasive tumor areas. Total
RNA was extracted using the High Pure FFPET RNA Isolation Kit (Roche)
following the manufacturer's protocol. RNA was hybridized overnight to
probes of the NanoString nCounter PanCancer Immune Profiling Panel with
Panel Plus customization (784 genes spanning canonical cancer pathways,
including MAPK, STAT, PI3K, RAS, Cell Cycle, Apoptosis, Hedgehog, Wnt,
DNA Damage Control, Transcriptional Regulation, Chromatin Modification,
and TGF-β). \cite{Cesano2015PanCancer,NanoStringPanCancerImmunePanel} purified, and digitally counted on the nCounter platform
(NanoString Technologies). Manufacturer controls were used to assess RNA
quality and run performance. Raw Reporter Code Count (RCC) files were
used for downstream analysis.

\subsubsection{Pre-processing and differential expression
analysis}\label{pre-processing-and-differential-expression-analysis}

Raw counts were normalized and transformed in R. Background correction
and positive/negative control normalization were applied as per
NanoString recommendations. Library size normalization and variance
modeling were performed with voom (limma), and log-counts per million
(logCPM) values were computed (edgeR). Differential expression was
assessed with linear modeling and empirical Bayes moderation (limma).
Multiple testing correction used the Benjamini--Hochberg false discovery
rate (FDR); unless stated otherwise, significance thresholds were FDR
\textless{} 0.05 and \textbar log2FC\textbar{} ≥ 1. Principal component
analysis (PCA) was performed on scaled logCPM matrices. Unsupervised
hierarchical clustering used Euclidean distances and Ward's linkage. All
scripts are available in the GitHub project repository.

\subsubsection{Machine-learning classification of
histology}\label{machine-learning-classification-of-histology}

Histologic class (PD vs WD) was modeled using logistic regression (LR)
and a random forest classifier (RFC) implemented in scikit-learn
(Python). Features comprised normalized transcript levels
(\textasciitilde800 genes). Data were stratified to preserve class
balance. Model evaluation used stratified 5-fold cross-validation,
reporting accuracy, balanced accuracy, precision/recall, F1, and
ROC-AUC. Hyperparameters were tuned via RandomizedSearchCV with internal
cross-validation. Confusion matrices and ROC/PR curves were generated
for interpretability.

\subsubsection{Feature importance and model
interpretability}\label{feature-importance-and-model-interpretability}

Global and local feature attributions were derived from: (i) LR
coefficients (L1/L2/elastic-net regularization), (ii) RFC Gini
importance, and (iii) SHAP (Shapley Additive exPlanations) values
computed on the RFC using mean absolute class-specific SHAP. Robust
markers were identified by intersecting features meeting pre-specified
thresholds ( \textbar β\textbar{} ≥ 0.04 for LR; Gini ≥ 0.001; mean SHAP
(PD class) ≥ 0.001). Gene-level summaries and per-sample SHAP plots are
provided in Supplementary materials.

\subsubsection{Pathway enrichment
analysis}\label{pathway-enrichment-analysis}

Pathway analysis was performed in Cytoscape (v3.10.1) using gene sets
from LR (positive/negative coefficients under L1/L2/elastic-net) and
from RFC SHAP/Gini (PD- and WD-enriched features analyzed separately).
Enrichment used pathway databases (e.g., WikiPathways/Reactome) and
network summarization (e.g., EnrichmentMap) with similarity metrics to
cluster related pathways. Reported statistics include similarity scores
and FDR-adjusted p-values. Cell-cycle modules (WP179, G1→S, G2/M, ATR)
and PI3K--Akt/RB pathways were prioritized based on effect size and
recurrence across models.

\subsubsection{Primary-site inference for tumors of unknown
origin}\label{primary-site-inference-for-tumors-of-unknown-origin}

For five tumors lacking annotated primaries, site inference used
k-nearest neighbors (K = 1) on the full logCPM feature space. Similarity
was quantified by Pearson correlation; distance was defined as 1 − r.
Each unknown was assigned the primary site of its nearest labeled
neighbor. Pairwise correlations were visualized in a heatmap with
nearest-neighbor links highlighted. For selected pairs, quantile--quantile
(QQ) plots visualized concordance. Sensitivity analyses varying K and
correlation/distance metrics are provided in Supplementary materials.

\subsubsection{External validation using DNA methylation
data}\label{external-validation-using-dna-methylation-data}

Independent validation was performed on public lung NEN methylation data
(Illumina Infinium MethylationEPIC; GSE211483). Raw IDAT files were
processed in R with minfi/ChAMP. QC included control probes and total
signal; probes with detection p \textgreater{} 0.01 in any sample, sex
chromosome probes, and probes with known SNPs at the CpG or extension
site were removed. Background correction and dye-bias normalization used
noob; BMIQ harmonized type I/II probe distributions. β-values were used
for visualization; M-values [log2(β/(1−β))] were used for statistical
testing where appropriate.

Unsupervised analyses (PCA and hierarchical clustering) were performed
on filtered β-matrices without variance preselection for the primary
figures; sensitivity analyses using variable CpGs were also evaluated.
Differentially methylated positions (DMPs) were identified using linear
models with empirical Bayes moderation and FDR control (typically FDR
\textless{} 0.05 and \textbar Δβ\textbar{} ≥ 0.2). Genomic context
enrichment (promoters, CpG islands, shores/shelves) was evaluated, and
Polycomb target--based stemness (pcgtAge) scores were computed. Where
matched RNA data (GSE211486) were available, methylation--expression
relationships were assessed by Wilcoxon tests with
Benjamini--Hochberg adjustment.

\subsubsection{Statistical analyses}\label{statistical-analyses}

All statistical analyses were performed in R (version ≥ 4.1) and Python
(version ≥ 3.10). Unless otherwise noted, two-sided tests were used with
FDR control by Benjamini--Hochberg. Continuous variables are summarized
as median (IQR) unless specified. Group comparisons for IHC H-scores
used Wilcoxon rank-sum tests with FDR adjustment; correlations used
Pearson's r. Plots were generated using ggplot2, ComplexHeatmap,
matplotlib, and seaborn. Reproducibility was supported by scripted
analyses, version-controlled code, and fixed random seeds where
applicable.

\subsubsection{Software and versions}\label{software-and-versions}

Key software included: R (≥ 4.1), Bioconductor (minfi, ChAMP, limma,
edgeR), Python (≥ 3.10), scikit-learn (≥ 1.3) for LR/RFC/CV, shap (≥
0.43) for SHAP, and Cytoscape (v3.10.1). Exact package versions and
command-line invocations are documented in the repository.
