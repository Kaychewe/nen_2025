\subsection{\texorpdfstring{Conclusions}{Conclusions}}\label{conclusions-1}

This study demonstrates that neuroendocrine neoplasms (NENs) can be
robustly stratified into well-differentiated NETs and poorly
differentiated NECs using transcriptomic data and interpretable machine
learning. By integrating logistic regression, random forest classifiers,
and SHAP-based feature attribution, we identified reproducible gene
signatures, including \emph{SFN}, \emph{FANCA}, \emph{HIST1H3B},
\emph{CHEK1}, and \emph{CAMK2B}, that discriminate histologic subtypes
with high accuracy. These markers reflect underlying biological
differences, such as cell cycle activation in PD-NECs and neuronal
identity in WD-NETs.

Importantly, we validated these findings across multiple levels: (1)
orthogonal protein expression by immunohistochemistry, (2) inference of
unknown tumor primaries via transcriptomic proximity, and (3)
independent external validation using DNA methylation data from a
separate lung NEN cohort. The concordant separation of subtypes across
expression, methylation, and machine learning predictions underscores
the robustness of these molecular signatures.

By combining predictive performance with biological interpretability,
our approach offers both diagnostic utility and mechanistic insight.
These results highlight the potential for AI-guided classifiers to
augment histopathological diagnosis, uncover new biomarkers, and resolve
ambiguous or mixed NEN cases. More broadly, this framework can be
adapted to other cancer types where histologic grading is subjective,
incomplete, or difficult to resolve.
